\documentclass[a4paper, 12pt]{article}
\usepackage{graphicx}
\usepackage[spanish, es-tabla]{babel}
\usepackage[T1]{fontenc}
\usepackage{listings}

\title{Documento de requerimientos}
\author{Campos Mora Luis Edel}

\begin{document}
\begin{figure}
\caption{logo universidad veracruzana}
\centering
\includegraphics[scale=1]{Logo-UV.jpg} 




\end{figure}
\begin{abstract}
Este es un documento de requerimientos del proyecto de la calculadora de IMC
\end{abstract}
Alcance del software
\begin{verse}	

	El próposito general de este proyecto, es aprobar con una buena calificación 	 
\end{verse}

Referencias
\begin{itemize}
\item Valencia, M. E. (1989). Guia para la preparación de un documento de requerimientos. Santiago de Cali, Valle del Cauca, Colombia.
\end{itemize}
Funcionalidades del proyecto
\begin{verse}
Este proyecto puede calcular el indice de masa corporal de una persona, de cualquier género
\end{verse}
Clases y características del usuario
\begin{verse}
\begin{enumerate}
\item Usuario general
\begin{itemize}
\item Cualquier persona que tenga el url de la aplicación web en Firebase
\end{itemize}
\end{enumerate}
\end{verse}
Entorno operativo
\begin{verse}
Cualquier computadora que cuente con acceso a internet
\end{verse}
Requerimientos funcionales
\begin{itemize}
\item El calculo de el IMC se debe de realizar de manera correcta
\item Debe de haber un campo para escribir la estatura
\item Debe de haber un campo para escribir el peso
\item Debe de haber un botón para realizar el calculo, para cuando los campos se han llenado de manera exitosa
\end{itemize}
Requerimientos de interfaces externas
	N/A
Requerimientos no funcionales
\begin{itemize}
\item Que el usuario sepa leer
\item Que el usuario sepa escribir
\end{itemize}
Glosario
\begin{itemize}
\item IMC: El índice de masa corporal es una razón matemática que asocia la masa y la talla de un individuo, ideada por el estadístico belga Adolphe Quetelet, por lo que también se conoce como índice de Quetelet.
\item Campo: Espacio para escribir
\item Internet: Internet es un neologismo del inglés que significa red informática descentralizada de alcance global. Se trata de un sistema de redes interconectadas mediante distintos protocolos que ofrece una gran diversidad de servicios y recursos, como, por ejemplo, el acceso a archivos de hipertexto a través de la web.

\item Firebase: Firebase es una plataforma para el desarrollo de aplicaciones web y aplicaciones móviles desarrollada por Google en 2014.
\item Aplicación Web: En la ingeniería de software se denomina aplicación web a aquellas herramientas que los usuarios pueden utilizar accediendo a un servidor web a través de internet o de una intranet mediante un navegador.
\end{itemize}
\end{document}


